%==================================================
%      PREAMBOLO e DICHIARAZIONI INIZIALI
%==================================================
\documentclass[10pt,oneside,a4paper]{article}

\usepackage[latin1]{inputenc} 
\usepackage[italian]{babel}
\usepackage{siunitx} %Inserisce automaticamente i dati con le unit�  di misura correttamente formattate del SI (utilizzo: \SI{0.82}{m^2}, in generale \SI{misura con il punto decimale}{unit�  di misura})
\sisetup{output-decimal-marker = {.}, separate-uncertainty = true, input-uncertainty-signs = \pm, detect-weight=true, detect-family=true} %per usare SI con il punto decimale
\usepackage{listings} %Per citare codice informatico formattandolo correttamente
\usepackage{amsmath}
\usepackage{graphicx}
\usepackage{geometry}
\usepackage{epigraph}
\usepackage{booktabs}	%tabelle migliorate
\usepackage{tablefootnote}	%note a pi� di pagina in tabella
\usepackage{threeparttable} %tabella con note a pi� di tabella
\usepackage{caption}	%descrizione per figure
\captionsetup{tableposition=top,figureposition=bottom,font=small} %setup descrizione
\usepackage{float}
\usepackage{esvect} %vettori
\usepackage{longtable} %tabelle lunghe
\usepackage[dvipsnames]{xcolor}
\definecolor{sepia}{HTML}{80002A}
\usepackage[colorlinks=true, citecolor=black, linkcolor=sepia, urlcolor=black]{hyperref}
\usepackage{mathrsfs}
%\usepackage[utf8]{inputenc}

\newcommand{\var}{\operatorname{var}}
\newcommand{\cov}{\operatorname{cov}}


\usepackage{listings} %Per inserire codice
\lstnewenvironment{codice_c}[1][]
{\lstset{basicstyle=\small\ttfamily, columns=fullflexible,
keywordstyle=\color{red}\bfseries, commentstyle=\color{blue},
language=C, basicstyle=\small,
numbers=left, numberstyle=\tiny,
stepnumber=2, numbersep=5pt, frame=shadowbox,  showstringspaces=false, #1}}{}

\setcounter{section}{-1}

%==================================================
%                  PRIMA PAGINA
%==================================================

\title{\textsc{Termometria e Calorimetria}}
\author{\small{G. Galbato Muscio} \and \small{L. Gravina} \and \small{L. Graziotto}}
\date{}

\begin{document}
	\begin{figure}
		\centering
		\includegraphics[scale=0.5, trim={2.8cm 8.9cm 0 9cm}, clip]{logo.png}
	\end{figure}
	\maketitle
	\begin{center} 
		\fbox{{\fontsize{12pt}{8mm}\textsc{Gruppo B}}} \\
		\vspace{1cm}
		\begin{tabular}{ccc}
			Esperienza di laboratorio && Consegna della relazione \\
			\emph{\small{25 ottobre 2017}} && \emph{\small{2 novembre 2017}} \\
		\end{tabular} 
		\vspace{0.5cm}
	\end{center}
\hrule
\vspace{0.5cm}
\begin{abstract}
Si determinano sperimentalmente il calore specifico di un campione ignoto, il calore latente di fusione del ghiaccio, la costante di tempo di un termometro a mercurio e si studia la perdita di calore di un calorimetro, mostrandone la distanza dal modello ideale di contenitore adiabatico.
\end{abstract}
\newpage
\tableofcontents %Indice
\listoftables %Indice delle tabelle
\listoffigures %Indice dei grafici

%==================================================
%         SCOPO E DESCRIZIONE DELL'ESPERIENZA
%==================================================
\section{Scopo e descrizione dell'esperienza}
\label{sec:descrizione}


%==================================================
%				APPARATO SPERIMENTALE
%==================================================		
\section{Apparato Sperimentale}
\subsection{Strumenti}
\label{subsec:strumenti}
\begin{itemize}
	\item 2 Calorimetri Dewar;
	\item Bilancia [portata: \SI{8000}{g}, risoluzione: \SI{0.1}{g}, incertezza: \SI{0.03}{g}];
	\item Cronometro [risoluzione: \SI{0.01}{s}, incertezza: \SI{0.003}{s}];
	\item Termometro a mercurio [risoluzione: \SI{1}{\degree C}, incertezza: \SI{0.3}{\degree C}];
	\item Termometro a mercurio [risoluzione: \SI{0.2}{\degree C}, incertezza: \SI{0.03}{\degree C}];
	\item Agitatore;
	\item Tappo per calorimetro.
\end{itemize}
\subsection{Campioni}
\begin{itemize}
	\item Bustine di plastica contenenti ghiaccio;
	\item Campione cilindrico di materiale metallico.
\end{itemize}
%==================================================
%            SEQUENZA OPERAZIONI SPERIMENTALI
%==================================================
\section{Sequenza Operazioni Sperimentali} 

\subsection{Misura del calore specifico del campione di metallo}
\label{subsec:calspec}

\subsection{Misura del calore latente di fusione del ghiaccio}
\label{subsec:ghiaccio}
\paragraph{Modello fisico ideale}
Il \emph{calore latente di fusione} $\Lambda_f $ � per definizione la quantit� di calore assorbita da una massa di ghiaccio unitaria durante il passaggio di stato da solido a liquido, al fine di rompere i legami tra le molecole d'acqua, che avviene ad una temperatura costante. Al fine di fornire una misurazione di questa grandezza, si misceleranno nel calorimetro una massa d'acqua calda $ M_{H_2O} $ alla temperatura $T_{H_2O}$ e una massa di ghiaccio $M_g$ alla temperatura prossima a \SI{0}{\degree C}: i due sistemi termodinamici interagiranno tra loro, il ghiaccio fonder� interamente e quindi la massa d'acqua complessiva si porter� alla temperatura di equilibrio $T_\text{eq}$. Utilizzando il Primo Principio della Termodinamica e approssimando il calorimetro come un contenitore adiabatico si ha
\begin{equation*}
\Delta U = Q_{H_2O} + Q_\text{g} - L_{H_2O} - L_\text{g};
\end{equation*}
poich� l'energia interna non varia tra l'inizio e la fine del processo e il sistema non compie lavoro si ha $L_{H_2O}=L_\text{g} = 0$ e dunque
\begin{equation*}
0 = Q_{H_2O} + Q_\text{g};
\end{equation*}
la quantit� di calore assorbita dal ghiaccio � data dalla somma tra il calore latente di fusione moltiplicato per la massa di ghiaccio e la quantit� di calore assorbita dalla massa di ghiaccio fusa, dunque di una massa d'acqua portata dalla temperatura di \SI{0}{\degree C} alla temperatura di equilibrio pari alla massa del ghiaccio introdotto allo stato solido. Dunque si ottiene
\begin{equation*}
c_{H_2O} M_{H_2O} (T_\text{eq}-T_{H_2O}) + c_{H_2O} M_\text{g} T_\text{eq} + \Lambda_f M_g = 0,
\end{equation*}
esplicitando per $\Lambda_f$,
\begin{equation}
\label{eq:lambda}
\Lambda_f  = \frac{c_{H_2O}}{M_g}\big[M_{H_2O}(T_{H_2O}-T_\text{eq}) - M_g T_\text{eq}\big]. 
\end{equation}
\paragraph{Procedura e presa dati}
Si � operato nel modo seguente:
\begin{enumerate}
\item Si pesa con la bilancia la massa del calorimetro con il tappo;
\item Si inserisce una massa d'acqua calda nel calorimetro, e lo si chiude con il tappo; 
\item Si pone il calorimetro sulla bilancia e si determina, per differenza con la pesata precedente, la massa d'acqua $M_{H_2O}$ introdotta;
\item Si misura con il termometro con divisione \SI{0.2}{\degree C} la temperatura $T_{H_2O}$ della massa d'acqua introdotta;
\item Si tara la bilancia e si inserisce rapidamente nel calorimetro una massa di ghiaccio $M_g$, misurata direttamente sulla bilancia, costituita da cubetti avvolti in un involucro di plastica; i cubetti si trovano in un bagno d'acqua alla temperatura di \SI{0.4}{\degree C}, perci� si assume che appena introdotti nel calorimetro comincino subito a compiere il passaggio di stato. Quindi si richiude il calorimetro con il tappo;
\item Si attende che il ghiaccio si sciolga e che la massa di acqua precedentemente solida si porti all'equilibrio termico con la massa d'acqua gi� presente nel calorimetro. Si misura con il termometro la temperatura di equilibrio $T_\text{eq}$.
\end{enumerate}
Per confrontare i valori ottenuti, l'esperienza � stata ripetuta due volte: i dati raccolti sono riportati nelle tabelle~\ref{tab:cal_lat1} e~\ref{tab:cal_lat2}.


\begin{table}
\centering
\begin{tabular}{c|c}
\multicolumn{1}{c}{Grandezza} & \multicolumn{1}{c}{Valore} \\
\hline
$M_g$			& \SI{81.5 \pm 0.1}{g}	\\
$M_{H_2O}$		& \SI{896.5 \pm 0.1}{g}	\\
$c_{H_2O}$		& \SI{4.18 \pm 0.01}{J/g/\degree C}\\
$T_{H_2O}$		& \SI{52.9 \pm 0.1}{\degree C}\\
$T_\text{eq}$	& \SI{37.4 \pm 0.1}{\degree C}\\
\hline
\end{tabular}
\caption{Dati raccolti per la misura del calore latente di fusione del ghiaccio, prima prova}
\label{tab:cal_lat1}
\end{table}
\begin{table}
\centering
\begin{tabular}{c|c}
\multicolumn{1}{c}{Grandezza} & \multicolumn{1}{c}{Valore} \\
\hline
$M_g$			& \SI{175.7 \pm 0.1}{g}	\\
$M_{H_2O}$		& \SI{512.4 \pm 0.1}{g}	\\
$c_{H_2O}$		& \SI{4.18 \pm 0.01}{J/g/\degree C}\\
$T_{H_2O}$		& \SI{50.0 \pm 0.1}{\degree C}\\
$T_\text{eq}$	& \SI{7.8 \pm 0.1}{\degree C}\\
\hline
\end{tabular}
\caption{Dati raccolti per la misura del calore latente di fusione del ghiaccio, seconda prova}
\label{tab:cal_lat2}
\end{table}

\paragraph{Analisi dati}
Dal momento che le incertezze sulle temperature sono molto maggiori rispetto a quelle sulle masse, si propagheranno solamente gli errori riguardanti le temperature. Inoltre, poich� una sola delle tre prove sperimentali ha avuto esito ragionevole, si adopereranno gli errori massimi anzich� quelli statistici, dunque si attribuir� \SI{0.1}{\degree C} quale incertezza del termometro, data dall'interpolazione a mezza tacca, senza considerare una distribuzione uniforme del valore vero all'interno di un intervallo pari a \SI{0.1}{\degree C}, che avrebbe prodotto una deviazione standard di $0.1 / \sqrt{12} = \SI{0.03}{\degree C}$. L'equazione che permette di calcolare l'incertezza su $\Lambda_f$ � dunque
\begin{equation}
\label{eq:err_lambda}
\begin{aligned}
\delta{\Lambda_f}^2 &= \Big(\frac{\partial \Lambda_f}{\partial T_{H_2O}}\Big)^2 \delta_{T_{H_2O}}^2 + \Big(\frac{\partial \Lambda_f}{\partial T_{\text{eq}}}\Big)^2 \delta_{T_{\text{eq}}}^2 \\
&= \Big(c_{H_2O}\frac{M_{H_2O}}{M_g}\Big)^2 \delta_{T_{H_2O}}^2 + \Big(c_{H_2O}\frac{M_{H_2O}+M_g}{M_g}\Big)^2 \delta_{T_{\text{eq}}}^2,
\end{aligned}
\end{equation}
dove con $\delta$ si � indicata l'incertezza massima. Sostituendo i dati raccolti nella~\ref{eq:lambda} e nella~\ref{eq:err_lambda}, si ottiene il valore per il calore latente di fusione del ghiaccio nella prima prova di
\begin{equation*}
\boxed{\bf{\Lambda_f = \SI{556 \pm 7}{J / g}}},
\end{equation*}
e nella seconda prova di
\begin{equation*}
\boxed{\bf{\Lambda_f = \SI{482 \pm 2}{J / g}}}.
\end{equation*}
Entrambi i valori sono notevolmente distanti dal valore vero di \SI{333.5}{J / g}, dunque si ritiene che la procedura di presa dati sia stata affetta da un errore sistematico, che ha influenzato le misure di temperatura. In primo luogo, si � osservato che il valore meno distante da quello reale � quello ricavato nel secondo tentativo: riducendo infatti la quantit� d'acqua e raddoppiando la massa di ghiaccio inserita, infatti, si � ottenuto un errore minore; confrontando con l'equazione~\ref{eq:err_lambda}, si verifica infatti una tendenza alla riduzione dell'incertezza sul valore del calore latente all'aumentare della massa di ghiaccio rispetto alla massa d'acqua. Tuttavia, bisogna tener conto del fatto che non vi � un limite nella riduzione della massa di acqua calda in favore di quella di ghiaccio, dato dal fatto che la prima deve riuscire a sciogliere per intero la seconda. A riprova di questo, si potrebbe ripetere l'esperienza introducendo masse uguali di ghiaccio e acqua calda, eventualmente quest'ultima ad una temperatura maggiore.

Un secondo ordine di errore si ritiene sia dato dal non aver considerato l'equivalente in massa del calorimetro $M_e$, ossia la misura di una quantit� d'acqua che, ipoteticamente posta nel calorimetro insieme con gli altri sistemi termodinamici, assorbirebbe per intero la quantit� di calore che � invece dispersa attraverso le pareti non idealmente adiabatiche del contenitore. L'equazione~\ref{eq:lambda}, tenendo conto di tale grandezza, diventa
\begin{equation*}
\Lambda_f  = \frac{c_{H_2O}}{M_g}\big[(M_{H_2O}+M_e)(T_{H_2O}-T_\text{eq}) - M_g T_\text{eq}\big]. 
\end{equation*}
� evidente, comunque, che tale correzione non farebbe altro che peggiorare il risultato gi� presentato, rendendolo ancora maggiore. Pertanto si ritiene che l'errore non risieda in questa omissione. Eventualmente, si ritiene che il calorimetro possa aver ceduto calore attraverso le pareti durante il processo, come studiato alla sezione~\ref{subsec:dispersione}; i tempi di esecuzione, tuttavia, sono risultati troppo brevi per poter apprezzare cali sensibili della temperatura di equilibrio dovuti alla non adiabaticit� delle pareti del calorimetro.

Terzo ordine di errore si ipotizza sia dato dal non aver reso uniforme la massa d'acqua una volta sciolto il ghiaccio, mescolando il sistema termodinamico con l'agitatore al fine di eliminare i rimanenti gradienti di temperatura, che potrebbero aver portato ad una lettura non definitiva della temperatura di equilibrio sul termometro. In una terza prova che qui non � discussa in quanto � risultata eccessivamente affetta da errore per portare un risultato comparabile con gli altri, si � osservato infatti che la temperatura di equilibrio continuava a scendere fino a raggiungere valori prossimi ai \SI{5}{\degree C}, non compatibili con la temperatura di equilibrio che ci si aspetterebbe calcolandola con il valore noto del calore latente di fusione. Si ipotizza pertanto che il sistema termodinamico sia molto instabile durante tutta la durata del processo di scioglimento del ghiaccio e del seguente raggiungimento della temperatura di equilibrio: in una futura ripetizione dell'esperienza, quindi, si ritiene sarebbe necessario attendere fino ad una eventuale risalita della temperatura del sistema, a causa della uniformazione della temperatura dell'acqua contenuta nel calorimetro.

Si riassumono pertanto le possibili migliorie all'esperimento, al fine di conseguire un risultato compatibile con il valore vero:
\begin{itemize}
\item Cambiamento della proporzione tra le masse d'acqua e di ghiaccio, con tentativo ponendo $M_g \simeq M_{H_2O}$;
\item Mescolamento migliore dei sistemi termodinamici all'interno del calorimetro;
\item Rimozione della plastica che avvolge i cubetti di ghiaccio, che pu� influire sul calore assorbito dallo stesso;
\item Attesa maggiore per la acquisizione della temperatura di equilibrio, che si ipotizza possa risalire dopo la discesa iniziale;
\item Inserimento nel calorimetro della massa di ghiaccio da sola, e quindi aggiunta successiva della massa d'acqua calda per mezzo dei fori presenti nel tappo del calorimetro, in modo da bilanciare meglio le masse e portare la temperatura di equilibrio ad un valore intermedio tra quello dell'acqua calda e quello del ghiaccio, analogamente al caso discusso nella sezione~\ref{subsec:calspec}; l'inserimento dell'acqua direttamente dal foro ridurrebbe inoltre le dispersioni di calore dovute alla rimozione del tappo.
\end{itemize}


\subsection{Determinazione della costante di tempo del termometro a mercurio}
\label{subsec:temptermo}

\subsection{Studio della dispersione di calore del calorimetro}
\label{subsec:dispersione}



%==================================================
%				    CONCLUSIONI
%==================================================
\section{Approfondimenti e considerazioni finali}




%==================================================
%       			APPENDICE
%==================================================

\section{Appendice: tabelle e grafici}


\end{document}
