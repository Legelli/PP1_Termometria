%==================================================
%      PREAMBOLO e DICHIARAZIONI INIZIALI
%==================================================
\documentclass[10pt,oneside,a4paper]{article}

\usepackage[latin1]{inputenc} 
\usepackage[italian]{babel}
\usepackage{siunitx} %Inserisce automaticamente i dati con le unit�  di misura correttamente formattate del SI (utilizzo: \SI{0.82}{m^2}, in generale \SI{misura con il punto decimale}{unit�  di misura})
\sisetup{output-decimal-marker = {.}, separate-uncertainty = true, input-uncertainty-signs = \pm, detect-weight=true, detect-family=true} %per usare SI con il punto decimale
\usepackage{listings} %Per citare codice informatico formattandolo correttamente
\usepackage{amsmath}
\usepackage{graphicx}
\usepackage{geometry}
\usepackage{epigraph}
\usepackage{booktabs}	%tabelle migliorate
\usepackage{tablefootnote}	%note a pi� di pagina in tabella
\usepackage{threeparttable} %tabella con note a pi� di tabella
\usepackage{caption}	%descrizione per figure
\captionsetup{tableposition=top,figureposition=bottom,font=small} %setup descrizione
\usepackage{float}
\usepackage{esvect} %vettori
\usepackage{longtable} %tabelle lunghe
\usepackage[dvipsnames]{xcolor}
\definecolor{sepia}{HTML}{80002A}
\usepackage[colorlinks=true, citecolor=black, linkcolor=sepia, urlcolor=black]{hyperref}
\usepackage{mathrsfs}
%\usepackage[utf8]{inputenc}

\newcommand{\var}{\operatorname{var}}
\newcommand{\cov}{\operatorname{cov}}


\usepackage{listings} %Per inserire codice
\lstnewenvironment{codice_c}[1][]
{\lstset{basicstyle=\small\ttfamily, columns=fullflexible,
keywordstyle=\color{red}\bfseries, commentstyle=\color{blue},
language=C, basicstyle=\small,
numbers=left, numberstyle=\tiny,
stepnumber=2, numbersep=5pt, frame=shadowbox,  showstringspaces=false, #1}}{}

\setcounter{section}{-1}

%==================================================
%                  PRIMA PAGINA
%==================================================

\title{\textsc{Termometria e Calorimetria}}
\author{\small{G. Galbato Muscio} \and \small{L. Gravina} \and \small{L. Graziotto}}
\date{}

\begin{document}
	\begin{figure}
		\centering
		\includegraphics[scale=0.5, trim={2.8cm 8.9cm 0 9cm}, clip]{logo.png}
	\end{figure}
	\maketitle
	\begin{center} 
		\fbox{{\fontsize{12pt}{8mm}\textsc{Gruppo B}}} \\
		\vspace{1cm}
		\begin{tabular}{ccc}
			Esperienza di laboratorio && Consegna della relazione \\
			\emph{\small{25 ottobre 2017}} && \emph{\small{2 novembre 2017}} \\
		\end{tabular} 
		
		\vspace{0.5cm}
		
	\end{center}
\hrule
\vspace{0.5cm}
\begin{abstract}
Si determinano sperimentalmente il calore specifico di un campione ignoto, il calore latente di fusione del ghiaccio, la costante di tempo di un termometro a mercurio e si studia la perdita di calore di un calorimetro, mostrandone la distanza dal modello ideale di contenitore adiabatico.
\end{abstract}
\newpage
\tableofcontents %Indice
\listoftables %Indice delle tabelle
\listoffigures %Indice dei grafici
\pagebreak
\section{Convenzioni e formule}
In questa relazione verranno usate le seguenti convenzioni:
\begin{enumerate}
	\item sar� usato il punto [ $.$ ] come separatore decimale;
	\item l'approssimazione decimale della cifra $5$ sar� fatta per eccesso;
	\item al fine di snellire la relazione e migliorarne la leggibilit�, riporteremo nel corpo del documento solamente le tabelle riepilogative e alcuni grafici, e dedicheremo un'appendice finale alle tabelle e ai grafici pi� dettagliati.
\end{enumerate}
Inoltre, si far� riferimento alle seguenti formule:
\begin{enumerate}
	\item media 
	\begin{equation}\label{eq:media}
	\bar{x} = \frac{1}{N}\sum_{i=1}^Nx_i;
	\end{equation}
	\item varianza
	\begin{equation}\label{eq:varianza}
	\sigma^2 = \frac{1}{N}\sum_{i=1}^N(x_i-\bar{x})^2;
	\end{equation}
	\item deviazione standard
	\begin{equation}\label{eq:deviazione}
	\sigma = \sqrt{\sigma^2}.
	\end{equation}	
\end{enumerate}
Per ricavare i parametri della regressione lineare si utilizzeranno le seguenti equazioni: 
\begin{enumerate}
	\item coefficiente angolare
	\begin{equation}\label{eq:coefficienteangolare}
	\hat{m} = \frac{\overline{xy}-\bar{x}\bar{y}}{\overline{x^2}\bar{x}^2};
	\end{equation}
	\item intercetta
	\begin{equation}\label{eq:intercetta}
	\hat{c} = \bar{y} - \hat{m} \cdot \bar{x};
	\end{equation}
	\item varianza del coefficiente angolare
	\begin{equation}\label{eq:varianzacoeff}
	\var[\hat{m}] = \frac{1}{\var[x]\cdot \sum_{i}\frac{1}{\sigma_{y_i}^2}};
	\end{equation}
	\item varianza dell'intercetta 
	\begin{equation}\label{eq:varianzainter}
	\var[\hat{c}] = \bar{x^2}\cdot \var[\hat{m}];
	\end{equation}
	\item covarianza 
	\begin{equation}\label{eq:cov}
	\cov[\hat{m}, \hat{c}] = - \frac{\bar{x}}{\var[\hat{m}]\cdot \sum_{i}\frac{1}{\sigma_{y_i}^2}}.
	\end{equation}
\end{enumerate}
%==================================================
%         SCOPO E DESCRIZIONE DELL'ESPERIENZA
%==================================================
\section{Scopo e descrizione dell'esperienza}
\label{sec:descrizione}


%==================================================
%				APPARATO SPERIMENTALE
%==================================================		
\section{Apparato Sperimentale}
\subsection{Strumenti}
\label{subsec:strumenti}
\begin{itemize}
	\item 2 Calorimetri;
	\item Bilancia [portata: \SI{8000}{g}, risoluzione: \SI{0.1}{g}, incertezza: \SI{0.03}{g}];
	\item Cronometro [risoluzione: \SI{0.01}{s}, incertezza: \SI{0.003}{s}];
	\item Termometro a mercurio [risoluzione: \SI{1}{�C}, incertezza: \SI{0.3}{�C}];
	\item Termometro a mercurio [risoluzione: \SI{0.2}{�C}, incertezza: \SI{0.03}{�C}];
	\item Agitatore;
	\item Tappo per calorimetro.
\end{itemize}
\subsection{Campioni}
\begin{itemize}
	\item Bustine di plastica contenenti ghiaccio;
	\item Campione cilindrico di materiale metallico.
\end{itemize}
%==================================================
%            SEQUENZA OPERAZIONI SPERIMENTALI
%==================================================
\section{Sequenza Operazioni Sperimentali} 

\subsection{Misura del calore specifico del campione di metallo}
\label{subsec:calspec}

\subsection{Misura del calore latente di fusione del ghiaccio}
\label{subsec:ghiaccio}

\subsection{Determinazione della costante di tempo del termometro a mercurio}
\label{subsec:temptermo}

\subsection{Studio della dispersione di calore del calorimetro}
\label{subsec:dispersione}



%==================================================
%				CONCLUSIONI
%==================================================
\section{Approfondimenti e considerazioni finali}




%==================================================
%       			APPENDICE
%==================================================

\section{Appendice: tabelle e grafici}


\end{document}
